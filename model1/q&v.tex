\documentclass[]{amsart} %size below in \usepackage[fontsize=14pt]{scrextend}
\usepackage{geometry}                % See geometry.pdf to learn the layout options. There are lots.
\geometry{letterpaper}                   % ... or a4paper or a5paper or ... 
%\geometry{landscape}                % Activate for for rotated page geometry
%\usepackage[parfill]{parskip}    % Activate to begin paragraphs with an empty line rather than an indent
\usepackage{graphicx}
\usepackage{amssymb}
\usepackage{epstopdf}
\usepackage[T1]{fontenc}
\usepackage[fontsize=14pt]{scrextend}
\DeclareGraphicsRule{.tif}{png}{.png}{`convert #1 `dirname #1`/`basename #1 .tif`.png}

\title{About quantities and values of firm capital goods: fixing the error in \emph{model1.4CapitalQ}}
\author{Anonymous}
\date{\today}                                           % Activate to display a given date or no date

\begin{document}
\maketitle

\section{Model's Inputs}

We have the value $K_{f,t}$ of the productive capital of the firm $f$ at time $t$. We subdivide it in $n$ constituent parts expressed as quotas $q_{f,i}$ with $\sum_{i=1}^{n} 
q_{f,i} = 1$. Quotas are specific of the ``sectorial class'' of the firm $f$.

We calculate---as sum of production costs---the prices of the $n$ types of capital productive goods, as $p_i$.

With 
\begin{equation}
k_{f,i} = q_{f,i}  K_{f,0}
\label{vAddenda}
\end{equation} 

as investment components of the firm $f$ in value, we obtain 

\begin{equation}
\hat{k}_{f,i} =  \frac{k_{f,i}}{p_i} 
\label{qAddenda}
\end{equation} 

as investment components of the firm $f$ in quantity.

$\hat{K_{f,0}}$ follows as:
\begin{equation}
\hat{K_{f,0}} = \sum_{i=1}^{n}  \hat{k}_{f,i} 
\label{hatK}
\end{equation} 


$\hat{K_{f,t}}$ is the productive capital of the firm $f$ at time $t$, in quantity; i.e., \emph{a bit strange sum of heterogeneous addenda}.

\section{Dynamic}

Adding the time dynamic, we have $\hat{K_{f,0}}$, obtained eliminating prices in  $K_{f,0}$, becoming $\hat{K_{f,t}}$ at time $t$.
To obtain $K_{f,t}$ we cannot use $q_{f,i}$ quotas, but $\hat{q}_{f,i}$ ones; with:

\begin{equation}
\hat{q}_{f,i}=\frac{\hat{k}_{f,i} }{\hat{K_{f,0}}}
\label{qQuotas}
\end{equation} 

We obtain:
\begin{equation}
K_{f,t}= \hat{K_{f,t}} \sum_{i=1}^{n}  \hat{q}_{f,i} p_i
\label{correctCalculation}
\end{equation} 

\section{An example}

With $K_{f,0}=1$ and $q_f=[\frac{1}{2},\frac{1}{2}]$,  $p=[1,2]$:

\begin{itemize}
\item From eq. (\ref{vAddenda}) follows $k_f = [\frac{1}{2} 1, \frac{1}{2} 1]$;

\item From eq. (\ref{qAddenda}) $\hat{k}_f = [\frac{1}{2}/1, \frac{1}{2}/2  ] = [\frac{1}{2} + \frac{1}{4}]$;

\item From eq. (\ref{hatK}) $\hat{K_{f,0}} = \frac{1}{2} + \frac{1}{4} = \frac{3}{4}$;

\item From eq. (\ref{qQuotas}) $\hat{q}_f = [\frac{1}{2}  / \frac{3}{4}, \frac{1}{4} / \frac{3}{4}] = [\frac{2}{3}, \frac{1}{3}]$;

\item From eq. ({correctCalculation}) $K_{f,t}= \frac{3}{4} [\frac{2}{3} 1 + \frac{1}{3} 2] = \frac{3}{4} \frac{4}{3} = 1$.

\end{itemize}

Using instead the wrong calculation way:
\begin{equation}
K_{f,t}= \hat{K_{f,t}} \sum_{i=1}^{n}  q_{f,i} p_i
\label{wrongCalculation}
\end{equation} 

we had obtained:
$K_{f,t}=  \frac{3}{4} (\frac{1}{2} 1 + \frac{1}{2} 2)=\frac{3}{4} \frac{3}{2} = \frac{9}{8} = 1.125$.

\section{The code}

In \verb|model1.4CapitalQ| we were implicitly using the miscalculation of eq. (\ref{wrongCalculation}) at the end of:

\scriptsize
\begin{verbatim}
    def produce(self,model)->tuple: 
\end{verbatim}
\normalsize

with:
\scriptsize
\begin{verbatim}
            self.desiredCapitalSubstitutions.append(desiredCapitalQsubstitutions*\
                    investmentComposition[self.sectorialClass][i]* model.investmentGoodPrices[i])
            self.requiredCapitalIncrement.append(requiredCapitalQincrement*\
                    investmentComposition[self.sectorialClass][i]* model.investmentGoodPrices[i])
\end{verbatim}
\normalsize

being \verb|investmentComposition| the initial quotas $q_{f,i}$.

Simplifying the eq. (\ref{correctCalculation}) for easier calculations we observe:

\begin{itemize}

\item using eq. (\ref{qQuotas}) in (\ref{correctCalculation}) we obtain:

\begin{equation}
K_{f,t}= \hat{K_{f,t}} \sum_{i=1}^{n}  \frac{\hat{k}_{f,i} }{\hat{K_{f,0}}} p_i = \frac{\hat{K_{f,t}}}{\hat{K_{f,0}}} \sum_{i=1}^{n}  \hat{k}_{f,i}  p_i
\label{firstSimp}
\end {equation}

\item using \ref{qAddenda} and then \ref{vAddenda} definitions, we have:

\begin{equation}
K_{f,t}= \frac{\hat{K_{f,t}}}{\hat{K_{f,0}}} \sum_{i=1}^{n}   \frac{k_{f,i}}{p_i}   p_i = \frac{\hat{K_{f,t}}}{\hat{K_{f,0}}} K_{f,0}
\label{secondSimp}
\end {equation}

\end{itemize}

Into the code we have to calculate the sum of \verb|desiredCapitalQsubstitutions| $+$ the sum \verb|requiredCapitalQincrement| and report the result to the initial capital expressed in quantity, e.g., \verb|self.capitalQ0| set in \verb|def settingCapitalQ| and never modified and apply the correction to obtain the new values of \verb|desiredCapitalSubstitutions| and \verb|requiredCapitalIncrement|, but this is only an apparent simplification, due to the presence of a lot of intermediate use of measure of the capital.as values or as quantities.

The actual simpler solution is to memorize the quotas of eq. (\ref{qQuotas}) within each firm and use them in the rows of code reported above, becoming:

\scriptsize
\begin{verbatim}
self.desiredCapitalSubstitutions.append(desiredCapitalQsubstitutions*\
         self.investmentCompositionQ[i]* model.investmentGoodPrices[i]
self.requiredCapitalIncrement.append(requiredCapitalQincrement*\
        self.investmentCompositionQ[i]* model.investmentGoodPrices[i])
\end{verbatim}
\normalsize
with the novelty of \verb|investmentCompositionQ[i]|.

\vspace*{1cm}

Using eqs. (\ref{qQuotas}) and (\ref{qAddenda}) we add the following code at the end of \verb|def settingCapitalQ|.

\scriptsize
\begin{verbatim}
self.investmentCompositionQ=[]
for i in range(len(params['investmentGoods'])):
      self.investmentCompositionQ.append(((investmentComposition[self.sectorialClass][i]*self.capital)/ \
             investmentGoodPrices[i])/self.capitalQ)
\end{verbatim}
\normalsize



\end{document}  
